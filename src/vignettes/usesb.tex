% scaleboot/src/vignettes/usesb.Rnw shimo@i.kyoto-u.ac.jp
%%\VignetteIndexEntry{Multiscale Bootstrap using Scaleboot Package}

\documentclass[a4paper]{amsart}
\usepackage{url}

\def\baselinestretch{1.0} 
\usepackage{Sweave}
\begin{document}
\input{usesb-concordance}


\title{Multiscale Bootstrap using Scaleboot Package}
\author{Hidetoshi Shimodaira}
\thanks{This document is a part of the {\tt scaleboot} package (Version 0.4-0 or newer). The source file is usesb.Rnw.
I thank Paul A. Sheridan for his comments to improve the earlier version of the manuscript.}
\address{Graduate School of Informatics, Kyoto University,
Yoshida Honmachi, Sakyo-ku, Kyoto, 606-8501, Japan}
\email{shimo@i.kyoto-u.ac.jp}


\maketitle

\section{Introduction}

{\tt scaleboot} is an add-on package for R. It is for calculating
approximately unbiased (AU) $p$-values for a general problem from a set
of multiscale bootstrap probabilities (BPs). Scaling is equivalent to
changing the sample size of a data in bootstrap resampling. We
compute BPs at several scales, from which a very accurate $p$-value is
calculated (Shimodaira 2002). This multiscale bootstrap method has
been implemented in {\tt CONSEL} (Shimodaira and Hasegawa 2001) for
phylogenetic inference and as the R add-on package {\tt pvclust}
(Suzuki and Shimodaira 2006) for hierarchical clustering. The point of
the {\tt scaleboot} package is to calculate an improved version of the
AU $p$-value that is justified even for hypotheses with nonsmooth
boundaries (Shimodaira 2008).

The basic usage of this package is illustrated in a simple example
below.  Then real applications in hierarchical clustering and
phylogenetic inference are shown later.

For the use of {\tt scaleboot}, Shimodaira (2008) may be referenced.

New update in 2018: Selective inference is now implemented.
The selective inference version of the approximately unbiased $p$-value
is denoted as SI in {\tt scaleboot} and {\tt pvclust}. The original approximately unbiased 
$p$-value, which is denoted as AU, is not designed for selective inference.
The bootstrap probability (BP) is also not designed for selective inference.
The theory and example of selective inference is described in Terada and Shimodaira (2017).

\section{Install}

{\tt scaleboot} is easily installed from CRAN online.  Windows users
can install the package by choosing ``scaleboot'' from the pull-down
menu.  Otherwise, run R on your computer and type
\begin{Schunk}
\begin{Sinput}
> install.packages("scaleboot")
\end{Sinput}
\end{Schunk}
You can also download the package file from the URL below and
install it manually.

\url{http://stat.sys.i.kyoto-u.ac.jp/prog/scaleboot/}


\section{Simple Example}

\subsection{Simulation Data}

We first generate a simulation data.
\begin{Schunk}
\begin{Sinput}
> simdata <- function(n,y,sd) {
+   m <- length(y)
+   x <- matrix(rnorm(m*n,0,sd),m,n)
+   t(x + (y - apply(x,1,mean)))
+ }
> X <- simdata(100,c(0,1,1,1,1,1,1,1,1,1),10)
> round(X[1:3,],3)
\end{Sinput}
\begin{Soutput}
       [,1]  [,2]   [,3]   [,4]   [,5]   [,6]   [,7]  [,8]    [,9]  [,10]
[1,] -6.558 0.925  0.312 10.095  2.894  3.519 -4.529 7.589  -6.712 -2.012
[2,] -0.638 0.573 -0.915  8.625  2.958  0.040 -2.599 5.552  -7.598 24.689
[3,] -5.917 7.251  3.721  8.961 -6.419 -4.051 -5.913 2.753 -10.037  4.057
\end{Soutput}
\begin{Sinput}
> y <- apply(X,2,mean)
> round(y,3)
\end{Sinput}
\begin{Soutput}
 [1] 0 1 1 1 1 1 1 1 1 1
\end{Soutput}
\end{Schunk}
The matrix $X=(x_{ij})$ above is of size $n\times m$ with $n=100$,
$m=10$. We consider $X$ as a data of sample size $n$, and rows
$x_i = (x_{i1},\ldots,x_{im})$, $i=1,\ldots,n$, are observations of a
random vector of $m$ dimensions.

\subsection{Null Hypothesis}

Let $\mu$ be the unknown population mean of the row vectors. An
estimate of $\mu$ is the sample average of the rows defined as $y =
\bar x = \tfrac{1}{n}\sum_{i=1}^n x_i $. Let $f(\mu)$ be a 0/1-valued
(or false/true valued) function of $\mu$. The null hypothesis we are
going to test is represented as $f(\mu)=1$. For example, $f(\mu)=1$ if
$\mu_1$ is the largest among $\mu_1,\ldots,\mu_m$, and $f(\mu)=0$
otherwise. This $f(\mu)$ is implemented as {\tt mc1(mu)} below.
\begin{Schunk}
\begin{Sinput}
> mc1 <- function(x) all(x[1] >= x[-1])
> mc1(y)
\end{Sinput}
\begin{Soutput}
[1] FALSE
\end{Soutput}
\end{Schunk}
Although $f(y)=0$ gives a rough idea whether $f(\mu)=1$, we want to
calculate a real number ranging between 0 and 1 which indicates the
possibility of $f(\mu)=1$.
This is what {\tt scaleboot} calculates as $p$-values. 

\subsection{Bootstrap Probabilities}

A naive way to calculate a $p$-value is by bootstrap resampling.  Let
$X^*=(x^*_{ij})$ be a bootstrap sample of $X$; each row $x^*_i$ is
obtained by resampling with replacement from the rows
$x_1,\ldots,x_n$.  Let $n'$ be the size of the resampling so that
$X^*$ is a matrix of size $n'\times m$. The bootstrap replicate is
$y^* = \bar x^* = \tfrac{1}{n'} \sum_{i=1}^{n'} x_i^*$. The following
code generates an $X^*$ with $n'=n$, and calculates $f(y^*)$. The
resampling is made via a weight vector {\tt w}; $w_i$ is the number of
times that $x_i$ is resampled in $X^*$.
\begin{Schunk}
\begin{Sinput}
> countw <- function(x,w,fn) {
+   y <- apply(w*x,2,sum)/sum(w)
+   fn(y)
+ }
> w <- as.vector(rmultinom(1,100,rep(1,100)))
> w
\end{Sinput}
\begin{Soutput}
  [1] 2 2 2 0 1 1 2 0 2 1 2 0 1 0 2 1 0 1 0 0 0 2 0 1 1 0 1 0 0 0 1 1 1 2 0 1 1
 [38] 2 1 0 1 1 1 0 2 1 1 1 0 0 1 1 0 2 1 0 2 1 1 1 3 1 0 2 1 3 0 0 0 0 1 1 2 1
 [75] 1 0 1 1 3 1 2 1 2 0 2 2 1 0 1 4 1 1 0 1 1 1 2 1 1 1
\end{Soutput}
\begin{Sinput}
> countw(X,w,mc1)
\end{Sinput}
\begin{Soutput}
[1] FALSE
\end{Soutput}
\end{Schunk}

Let $B$ be the number of bootstrap samples we will generate, and
$y^*_1,\ldots, y^*_B$ be the bootstrap replicates. Typically,
$B=10,000$. The BP is computed as $\sum_{i=1}^B f(y^*_i)/B$, where the
ordinary BP uses $n'=n$.  Since first introduced by Felsenstein
(1985), it has been widely used as a $p$-value, but the bias is in
fact rather large.


\subsection{$P$-value Calculation}

{\tt scaleboot} calculates corrected $p$-values for improving BPs.
First load the package by
\begin{Schunk}
\begin{Sinput}
> library(scaleboot)
\end{Sinput}
\end{Schunk}

Below, {\tt sa} specifies the scales, and {\tt nb} specifies
$B$ for each scale, so that $10,000\times13=130,000$ bootstrap samples
are generated internally.  It takes a few minutes on a pc.
\begin{Schunk}
\begin{Sinput}
> sa <- 9^seq(-1,1,length=13)
> nb <- 10000
> X.sb <- scaleboot(X,nb,sa,countw,mc1)
\end{Sinput}
\end{Schunk}
The main result (for $k=3$) is shown by
\begin{Schunk}
\begin{Sinput}
> summary(X.sb)  # k = 3 (default)
\end{Sinput}
\begin{Soutput}
Raw Bootstrap Probability (scale=1) : 0.93 (0.10)

Hypothesis: null 

Corrected P-values for Models (percent,Frequentist):
       k.3          sk.3         beta0       beta1       aic     weight 
sing.3 37.68 (2.94) 42.97 (2.83) 1.64 (0.02) 0.68 (0.01)  -12.35 100.00 
poly.3 17.23 (1.26) 22.84 (1.47) 1.67 (0.03) 0.59 (0.02)   15.02        
poly.2  6.34 (0.34)  9.70 (0.49) 1.92 (0.02) 0.39 (0.01)  177.96        
poly.1  0.07 (0.00)  0.13 (0.01) 3.21 (0.02) 0.00 (0.00) 3534.85        

Best Model:  sing.3 

Corrected P-values by the Best Model and by Akaike Weights Averaging:
        k.3          sk.3         beta0       beta1       
best    37.68 (2.94) 42.97 (2.83) 1.64 (0.02) 0.68 (0.01) 
average 37.68 (2.94) 42.97 (2.83) 1.64 (0.02) 0.68 (0.01) 
\end{Soutput}
\end{Schunk}
The selective inference $p$-values (SI) are denoted as {\tt sk.3},
and the non-selective AU $p$-values are denoted as {\tt k.3}.
We can also see additional results by specifying $k$,
\begin{Schunk}
\begin{Sinput}
> summary(X.sb, k=1:3)  # k = 1, 2, 3
\end{Sinput}
\end{Schunk}
{\tiny
\begin{Schunk}
\begin{Soutput}
Raw Bootstrap Probability (scale=1) : 0.93 (0.10)

Hypothesis: null 

Corrected P-values for Models (percent,Frequentist):
       k.1         k.2          k.3          sk.1        sk.2         sk.3         beta0       beta1       aic     weight 
sing.3 1.04 (0.04) 16.89 (0.82) 37.68 (2.94) 2.08 (0.09) 22.50 (0.97) 42.97 (2.83) 1.64 (0.02) 0.68 (0.01)  -12.35 100.00 
poly.3 1.20 (0.05) 14.18 (0.93) 17.23 (1.26) 2.39 (0.10) 19.60 (1.15) 22.84 (1.47) 1.67 (0.03) 0.59 (0.02)   15.02        
poly.2 1.03 (0.05)  6.34 (0.34)  6.34 (0.34) 2.06 (0.09)  9.70 (0.49)  9.70 (0.49) 1.92 (0.02) 0.39 (0.01)  177.96        
poly.1 0.07 (0.00)  0.07 (0.00)  0.07 (0.00) 0.13 (0.01)  0.13 (0.01)  0.13 (0.01) 3.21 (0.02) 0.00 (0.00) 3534.85        

Best Model:  sing.3 

Corrected P-values by the Best Model and by Akaike Weights Averaging:
        k.1         k.2          k.3          sk.1        sk.2         sk.3         beta0       beta1       
best    1.04 (0.04) 16.89 (0.82) 37.68 (2.94) 2.08 (0.09) 22.50 (0.97) 42.97 (2.83) 1.64 (0.02) 0.68 (0.01) 
average 1.04 (0.04) 16.89 (0.82) 37.68 (2.94) 2.08 (0.09) 22.50 (0.97) 42.97 (2.83) 1.64 (0.02) 0.68 (0.01) 
\end{Soutput}
\end{Schunk}
}

A class of AU $p$-values $p_k$ indexed by $k=1,2,3$, are calculated,
and they are labelled as {\tt k.1}, {\tt k.2}, and {\tt k.3}.  The
$p$-values are shown in percent, and the standard errors are given in
parentheses.  We should look at the row of {\tt average} (the bottom
line), and we can ignore the other rows.  $p_1\approx 1\%$ corresponds
to the ordinary BP, and $p_2\approx 18\%$ corresponds to the AU
$p$-value of Shimodaira (2002).  What we recommend to use here is
$p_3\approx40\%$; this is the AU $p$-value of Shimodaira (2008).  For
this particular example, the common practice for calculating a
$p$-value is to use the multiple comparisons method. If it is applied
to $y$, the $p$-value is $p=66\%$, which is rather close to $p_3$ in
our example, whereas $p_1$ is obviously too small.

Internally, as explained in the next section, several models are
fitted to the observed bootstrap probabilities. They are sorted in
increasing order of AIC, and the best model is {\tt sing.3}. The row
of {\tt sing.3} is duplicated in the row {\tt best}, two lines from
the bottom. The Akaike weights $\propto \exp(-{\rm AIC}/2)$ are also
computed for models and the $p$-values are averaged using the weights
so that we get the row of {\tt average}, the bottom line. Typically,
the two bottom lines are almost identical, and {\tt average} is
regarded as a smoothed version of {\tt best} for small changes in
data.  I recommend to use {\tt average} instead of {\tt best}.

\subsection{Internal Steps}

We consider the following three steps (i)-(iii).
Internally, the {\tt scaleboot} function (i) performs the multiscale
bootstrap, and (ii) estimates coefficients for candidate models. Then
the {\tt summary} method (iii) calculates the corrected $p$-values.
These steps are explained below.

The results of steps (i) and (ii) are shown here.
\begin{Schunk}
\begin{Sinput}
> X.sb
\end{Sinput}
\begin{Soutput}
Multiscale Bootstrap Probabilities (percent):
1    2    3    4    5    6    7    8    9    10   11   12   13   
0.00 0.01 0.05 0.12 0.29 0.68 0.93 1.57 2.21 2.77 3.40 3.94 4.69 

Numbers of Bootstrap Replicates:
1     2     3     4     5     6     7     8     9     10    11    12    13    
10000 10000 10000 10000 10000 10000 10000 10000 10000 10000 10000 10000 10000 

Scales (Sigma Squared):
1      2      3      4      5      6      7 8     9     10   11    12   13    
0.1111 0.1603 0.2309 0.3333 0.4808 0.6944 1 1.449 2.083 3.03 4.348 6.25 9.091 

Coefficients:
       beta0           beta1           beta2            
sing.3 1.1518 (0.1347) 1.1601 (0.1401)  0.8332 (0.1221) 
poly.3 1.6337 (0.0284) 0.6569 (0.0210) -0.0318 (0.0024) 
poly.2 1.9212 (0.0219) 0.3943 (0.0069)                  
poly.1 3.2056 (0.0182)                                  

Model Fitting:
       rss     df pfit   aic     
sing.3    7.65 10 0.6629  -12.35 
poly.3   35.02 10 0.0001   15.02 
poly.2  199.96 11 0.0000  177.96 
poly.1 3558.85 12 0.0000 3534.85 

Best Model:  sing.3 
\end{Soutput}
\begin{Sinput}
> 
\end{Sinput}
\end{Schunk}

The results of (i) are the BPs for the 13 scales shown at first. Let
$\alpha_{\sigma^2}$ denote the BP at scale $\sigma^2$. Each BP is
calculated from 10,000 bootstrap samples of size $n'$ as the frequency
of observing $f(y^*)=1$. In {\tt scaleboot}, $n'$ is {\tt
round(n/sa[i])}, for $i=1,\ldots,13$. Then, the scale is recalculated
as $\sigma^2=n/n'$ for taking account of the discreteness. 

Step (ii) is performed by the {\tt sbfit} function called internally
from the {\tt scaleboot} function for fitting parametric models to
observed $\alpha_{\sigma^2}$'s.  By default, four models are
considered as candidates; {\tt poly.1}, {\tt poly.2}, {\tt poly.3},
and {\tt sing.3}. Each of these models is denoted as
$\psi(\sigma^2|\beta)$. Let $z_{\sigma^2} = \Phi^{-1}( 1 -
\alpha_{\sigma^2})$ be the bootstrap $z$-value at scale $\sigma^2$,
where $\Phi^{-1}(p)$={\tt qnorm(p)}.  We work on $\sigma
z_{\sigma^2}(y)$, which may be called a normalized bootstrap
$z$-value.  Considering $\sigma z_{\sigma^2}$ as a function of
$\sigma^2$, the coefficient vector $\beta$ is estimated by fitting
$\sigma z_{\sigma^2} = \psi(\sigma^2 | \beta)$.  Let $\hat\beta$
denote the estimated value; the details of fitting $\hat\beta$ are
explained later.  We may choose the model which minimizes AIC
value. The fitted curves are shown (Fig.~\ref{fig:diag1f}) by plotting
$\psi(\sigma^2|\hat\beta)$ as
\begin{Schunk}
\begin{Sinput}
> plot(X.sb,legend="topleft")
\end{Sinput}
\end{Schunk}
\begin{figure}
\begin{center}
\includegraphics{usesb-013}
\caption{Model Fitting}\label{fig:diag1f}
\end{center}
\end{figure}
The same plot but in other variables can be shown
(Fig.~\ref{fig:diag1f2}) by, for example,
\begin{Schunk}
\begin{Sinput}
> plot(X.sb,xval="sigma",log="x",yval="pvalue",legend="topleft")
\end{Sinput}
\end{Schunk}
\begin{figure}
\begin{center}
\includegraphics{usesb-015}
\caption{Model Fitting ($x=\log \sigma$, $y=\alpha_{\sigma^2}$)}
\label{fig:diag1f2}
\end{center}
\end{figure}



{\tt poly.k} model is specified as a polynomial of $\sigma^2$;
$\psi(\sigma^2|\beta) = \sum_{j=0}^{k-1} \beta_j \sigma^{2j}$ for
$k\ge1$.  {\tt sing.k} model is specified as $\psi(\sigma^2|\beta) =
\beta_0 + \sum_{j=1}^{k-2} \beta_j \sigma^{2j}/(1 +
\beta_{k-1}(\sigma-1))$ for $k\ge3$, where $0\le\beta_{k-1}\le1$. The
number {\tt k} for each model denotes the number of coefficients in
$\beta$.

The details of model fitting are as follows. Let $B_i$ and $C_i$ be
the number of replicates and the observed number of times that
$f(y^*)=1$, respectively, for the bootstrap resampling of scale
$\sigma^2_i$, $i=1,\ldots,S$. Since each $C_i$ is binomially
distributed, the log-likelihood is
\[
\ell(\beta) = \sum_{i=1}^S \Bigl\{
C_i \log \Phi(-\psi(\sigma^2_i|\beta)/\sigma_i)+
(B_i - C_i) \log \Phi(\psi(\sigma^2_i|\beta)/\sigma_i)
\Bigr\},
\]
where $\Phi(q)$={\tt pnorm(q)}. The estimate $\hat\beta$ is obtained
by maximizing $\ell(\beta)$ numerically.  The goodness of fit is
measured by the difference of AIC values between the specified model
and an unconstrained binomial model;
\[
{\rm AIC} = ( -2\ell(\hat\beta)+2k ) - (-2\hat\ell + 2S),
\]
where $\hat\ell = \sum_{i=1}^S ( C_i\log(C_i/B_i) +
(B_i-C_i)\log(1-C_i/B_i)) $.


Step (iii) is performed by the the {\tt summary} method as already
mentioned. The first line shows the ``raw'' BP $\alpha_1$ (the BP
obtained from the ordinary bootstrap resampling).  The main results
are the corrected $p$-values, which follow next. For each model, we
calculate $q_k$, $k=1,2,3$, by
\[
q_k = \sum_{j=0}^{k-1} \frac{(-1-\sigma_0^2)^j}{j!}
\frac{\partial^j \psi(\sigma^2|\hat\beta)}{\partial (\sigma^2)^j}
\Bigr|_{\sigma_0^2}.
\]
Then the corrected $p$-values are calculated by $p_k=1-\Phi(q_k)$.  By
default $\sigma_0^2=1$. The calculation of $q_k$ is interpreted as
extrapolation of $\sigma z_{\sigma^2}$ to $\sigma^2=-1$ by using the
first $k$ terms of the Taylor series. According to the theory of
Shimodaira (2008), the unbiased $p$-value is, if it exists, obtained
by taking the limit $k\to\infty$. The extrapolated curves are shown
(Fig.~\ref{fig:diag1s}) by
\begin{Schunk}
\begin{Sinput}
> plot(summary(X.sb),legend="topleft")
\end{Sinput}
\end{Schunk}
\begin{figure}
\begin{center}
\includegraphics{usesb-017}
\caption{Extrapolation}\label{fig:diag1s}
\end{center}
\end{figure}


\section{Hierarchical Clustering}

\subsection{Pvclust Package}

The {\tt scaleboot} package includes an interface for the {\tt
pvclust} package (Suzuki and Shimodaira 2006).  We use {\tt pvclust}
to calculate multiscale BPs for clusters by bootstrapping hierarchical
clustering, from which we calculate an improved version of AU
$p$-values using {\tt scaleboot}. See {\tt help(lung73)} for further
details of the following example.

\subsection{Using Pvclust}

This example uses the {\tt lung} dataset (Garber et al.~2001) included
in {\tt pvclust}.  It is a DNA microarray data of 73 lung tissues
(arrays) with 916 observations of genes.  To draw dendrograms in terms
of the arrays, we resample genes in our analysis; this may be
interpreted as assessing the uncertainty due to the variability of
genes.  The function {\tt pvclust} first obtains a dendrogram by a
hierarchical clustering method, and then calculates the multiscale BPs
for each cluster of the dendrogram.
\begin{Schunk}
\begin{Sinput}
> library(pvclust)
> data(lung)
> sa <- 9^seq(-1,1,length=13)
> nb <- 10000
> lung73.pvclust <- pvclust(lung,r=1/sa,nboot=nb)
\end{Sinput}
\end{Schunk}
The above code may take a day, so it would be a good idea to run
with nb=1000 so that it would run 10 times faster. However, nb=1000
should be used just for checking the program, and nb=10,000 (at least)
is recommended for publishing the results.

\subsection{Model Fitting}

We next apply the {\tt sbfit} function of {\tt scaleboot} to the
multiscale BPs. For each cluster of the dendrogram, parametric models
are fitted to the BPs.
\begin{Schunk}
\begin{Sinput}
> library(scaleboot)
> lung73.sb <- sbfit(lung73.pvclust)
\end{Sinput}
\end{Schunk}

\subsection{Lung73 Dataset}

The results of the previous two sections ({\tt lung73.pvclust} and
{\tt lung73.sb}) are in fact stored in the {\tt lung73} dataset of {\tt
scaleboot}. For users who want to try the examples, just type as follows.
\begin{Schunk}
\begin{Sinput}
> library(scaleboot)
> data(lung73)
\end{Sinput}
\end{Schunk}

We have used a cluster computer of 40 cpus for parallel
computing using the {\tt snow} package.
The following code may run in under an hour.
\begin{Schunk}
\begin{Sinput}
> library(snow)
> cl <- makeCluster(40)
> library(pvclust)
> data(lung)
> sa <- 9^seq(-1,1,length=13)
> nb <- 10000
> lung73.pvclust <- parPvclust(cl,lung,r=1/sa,nboot=nb)
> library(scaleboot)
> lung73.sb <- sbfit(lung73.pvclust,cluster=cl)
\end{Sinput}
\end{Schunk}

\subsection{$P$-value Calculation}

To calculate AU $p$-values ($p_3$) from {\tt lung73.sb} and write them
back to {\tt lung73.pvclust}, we do
\begin{Schunk}
\begin{Sinput}
> lung73.k3 <- sbpvclust(lung73.pvclust,lung73.sb)
\end{Sinput}
\end{Schunk}
To see the results, we simply plot the dendrogram
(Fig.~\ref{fig:lung73tree}) by
\begin{Schunk}
\begin{Sinput}
> library(pvclust)
> plot(lung73.k3, cex=0.5, cex.pv=0.7)
> pvrect(lung73.k3, pv="si") # find clusters with p>0.95. Now use SI instead of AU.
\end{Sinput}
\end{Schunk}
\begin{figure}
\begin{center}
\includegraphics{usesb-024}
\caption{Dendrogram of lung73 dataset ($k=3$)}\label{fig:lung73tree}
\label{fig:lung73-pvclust}
\end{center}
\end{figure}
There three types of $p$-values are printed at each edge of the cluster. SI is the selective inference version of approximately unbised $p$-value, which is newly introduced in Terada and Shimodaira (2017). AU is the non-selective inference version of approximately unbiased $p$-value, which has been used since Shimodaira (2002). BP is the original bootstrap probability. SI, AU, BP are denoted as {\tt sk.3}, {\tt k.3}, {\tt k.1}, respectively, in tables generated by scaleboot for $k=3$.
When you know the cluster before looking at the data (such as cluster id=31, the left most cluster in Fig.~\ref{fig:lung73-pvclust}; it consists of normal cell and fetal cell, so no lung cancer at all), you may use AU.
However, you should use SI for most of the clusters when they are identified by looking at the tree.

To calculate $p_2$ instead of $p_3$, specify {\tt k=2},
\begin{Schunk}
\begin{Sinput}
> lung73.k2 <- sbpvclust(lung73.pvclust,lung73.sb, k=2)
\end{Sinput}
\end{Schunk}

\subsection{Diagnostics of Fitting}

The fitted curves are drawn by the plot method. For node 67, say, a
plot with legend is obtained (Fig.~\ref{fig:lungplot67}) by
\begin{Schunk}
\begin{Sinput}
> plot(lung73.sb[[67]],legend="topleft")
\end{Sinput}
\end{Schunk}
\begin{figure}
\begin{center}
\includegraphics{usesb-027}
\caption{Model fitting for node 67}\label{fig:lungplot67}
\end{center}
\end{figure}
All the calculated $p$-values for node 67 are given by
\begin{Schunk}
\begin{Sinput}
> summary(lung73.sb[[67]])
\end{Sinput}
\begin{Soutput}
Raw Bootstrap Probability (scale=1) : 3.33 (0.18)

Hypothesis: null 

Corrected P-values for Models (percent,Frequentist):
       k.3          sk.3         beta0       beta1       aic      weight 
sing.3 94.94 (0.18) 97.42 (0.09) 0.54 (0.01) 1.29 (0.01)    48.38 100.00 
poly.3 92.39 (0.29) 96.92 (0.14) 0.33 (0.00) 1.41 (0.01)   462.20        
poly.2 83.50 (0.32) 92.71 (0.19) 0.31 (0.00) 1.28 (0.01)  1307.79        
poly.1 18.54 (0.10) 37.08 (0.21) 0.89 (0.00) 0.00 (0.00) 52893.32        

Best Model:  sing.3 

Corrected P-values by the Best Model and by Akaike Weights Averaging:
        k.3          sk.3         beta0       beta1       
best    94.94 (0.18) 97.42 (0.09) 0.54 (0.01) 1.29 (0.01) 
average 94.94 (0.18) 97.42 (0.09) 0.54 (0.01) 1.29 (0.01) 
\end{Soutput}
\end{Schunk}
The extrapolation using the best models (averaged by the Akaike
weights) is shown (Fig.~\ref{fig:lungext67}) by
\begin{Schunk}
\begin{Sinput}
> plot(summary(lung73.sb[[67]]),legend="topleft")
\end{Sinput}
\end{Schunk}
\begin{figure}
\begin{center}
\includegraphics{usesb-030}
\caption{Extrapolation for node 67}\label{fig:lungext67}
\end{center}
\end{figure}

For a set of nodes, $p$-values are given by
\begin{Schunk}
\begin{Sinput}
> summary(lung73.sb[c(62,67,69,71)])
\end{Sinput}
\end{Schunk}
{\small
\begin{Schunk}
\begin{Soutput}
Corrected P-values by Akaike Weights Averaging (percent,Frequentist):
   raw          k.3          sk.3         beta0        beta1       hypothesis  model  weight 
62 95.67 (0.20) 98.40 (0.14) 96.21 (0.27) -1.95 (0.02) 0.21 (0.01) alternative poly.3  99.62 
67  3.33 (0.18) 94.94 (0.18) 97.42 (0.09)  0.54 (0.01) 1.29 (0.01) null        sing.3 100.00 
69 29.28 (0.46) 75.84 (0.34) 10.04 (0.40) -0.08 (0.00) 0.61 (0.01) alternative poly.3  43.83 
71 25.11 (0.43) 85.99 (0.27) 26.43 (0.47) -0.19 (0.00) 0.83 (0.01) alternative poly.3  98.47 
\end{Soutput}
\end{Schunk}
}
Also plots are shown (Fig.~\ref{fig:lung73nodes}) by
\begin{Schunk}
\begin{Sinput}
> plot(lung73.sb[c(62,67,69,71)])
\end{Sinput}
\end{Schunk}
\begin{figure}
\begin{center}
\includegraphics{usesb-034}
\caption{Model fitting for a set of nodes} \label{fig:lung73nodes}
\end{center}
\end{figure}

\section{Phylogenetic Inference}

\subsection{CONSEL Software}

{\tt scaleboot} has a front end for phylogenetic inference, and it may
eventually replace the {\tt CONSEL} software (Shimodaira and Hasegawa
2001) for testing phylogenetic trees. Currently, {\tt scaleboot} does
not have a method for converting files obtained from other commonly
used phylogenetic software packages, and so we must use {\tt CONSEL}
for this purpose before applying {\tt scaleboot} to calculate an
improved version of AU $p$-values for trees and edges.  See {\tt
help(mam15)} for further details of the following example.

\subsection{Mammal Dataset}

We work on an example of phylogenetic analysis of six mammal species:
Homo sapiens (human), Phoca vitulina (harbor seal), Bos taurus (cow),
Oryctolagus cuniculus (rabbit), Mus musculus (mouse), Didelphis
virginiana (opossum). The dataset was originally used in Shimodaira
and Hasegawa (1999). 

For Unix users, download {\tt mam15-files.tgz}, and for Windows users
download {\tt mam15-files.zip}.  The details of dataset files are as
follows.  {\tt mam15.aa}: amino acid sequences ($n=3414$) of mtDNA for
the six mammals.  {\tt mam15.ass}: association vectors for edges and
trees.  {\tt mam15.lnf}: site-wise log-likelihood values (output from
PAML).  {\tt mam15.log}: detailed information for the associations.
{\tt mam15.mt}: site-wise log-likelihood values (output from seqmt).
{\tt mam15.tpl}: 15 tree topologies.

\subsection{Likelihood Calculation of Trees}

The main body of the dataset is the amino acid sequences ({\tt mam15.aa}).
We  consider $m=15$ tree topologies of the six mammals ({\tt mam15.tpl});
\begin{verbatim}  
((Homsa,(Phovi,Bosta)),Orycu,(Musmu,Didvi)); t1
(Homsa,Orycu,((Phovi,Bosta),(Musmu,Didvi))); t2
(Homsa,((Phovi,Bosta),Orycu),(Musmu,Didvi)); t3
(Homsa,(Orycu,Musmu),((Phovi,Bosta),Didvi)); t4
((Homsa,(Phovi,Bosta)),(Orycu,Musmu),Didvi); t5
(Homsa,((Phovi,Bosta),(Orycu,Musmu)),Didvi); t6
(Homsa,(((Phovi,Bosta),Orycu),Musmu),Didvi); t7
(((Homsa,(Phovi,Bosta)),Musmu),Orycu,Didvi); t8
(((Homsa,Musmu),(Phovi,Bosta)),Orycu,Didvi); t9
(Homsa,Orycu,(((Phovi,Bosta),Musmu),Didvi)); t10
(Homsa,(((Phovi,Bosta),Musmu),Orycu),Didvi); t11
((Homsa,((Phovi,Bosta),Musmu)),Orycu,Didvi); t12
(Homsa,Orycu,(((Phovi,Bosta),Didvi),Musmu)); t13
((Homsa,Musmu),Orycu,((Phovi,Bosta),Didvi)); t14
((Homsa,Musmu),((Phovi,Bosta),Orycu),Didvi); t15
\end{verbatim}
The maximum likelihood estimates for these trees are calculated by
PAML (Yang 1997).  Let $x_{ij}$ be the site-wise log-likelihood for
sites $i=1,\ldots,n$, and trees $j=1,\ldots,m$. The log-likelihood of
tree-$j$ is $\sum_{i=1}^n x_{ij}$. A large $n$ justifies the central
limit theorem for $y=\bar x$, and allows us to resample $x_{ij}$
directly without recalculation of the maximum likelihood
estimates. The matrix $X=(x_{ij})$ is produced by PAML and stored in
{\tt mam15.lnf}.  It is converted by CONSEL to a simpler format and
stored in {\tt mam15.mt}. The command is

\verb!seqmt --paml mam15.lnf!


\subsection{$P$-value Calculation for Trees}

The AU $p$-values for trees are calculated simply by
\begin{Schunk}
\begin{Sinput}
> library(scaleboot)
> mam15.mt <- read.mt("mam15.mt")
> mam15.trees <- relltest(mam15.mt)
> summary(mam15.trees)
\end{Sinput}
\end{Schunk}
The {\tt relltest} function above may take a half hour. The next
section can be skipped if only tree selection is of interest.

\subsection{$P$-value Calculation for Clusters}

We can also calculate AU $p$-values for clusters (edges) of trees.  We
have to know, for each cluster, in which of the 15 trees it is
included.  The file {\tt mam15.ass} has this information, which was
generated using CONSEL by the command 

\verb!treeass --outgroup 6 mam15.tpl > mam15.log!

It also produces {\tt mam15.log} for human readable information.  A
part of {\tt mam15.log} is as follows.
\begin{verbatim}
# leaves: 6
6
  1 Homsa
  2 Phovi
  3 Bosta
  4 Orycu
  5 Musmu
  6 Didvi

# base edges: 10
10 6
          
    123456
  1 +++---  ;
  2 ++++--  ;
  3 +--+--  ;
  4 -+++--  ;
  5 ---++-  ;
  6 +--++-  ;
  7 -++++-  ;
  8 +++-+-  ;
  9 +---+-  ;
 10 -++-+-  ;
\end{verbatim}
The clusters (edges) defined above are named e1,...,e10. For example,
e1 = {\tt +++---} = (Homsa, Phovi, Bosta).

The AU $p$-values for clusters as well as trees are calculated simply by
\begin{Schunk}
\begin{Sinput}
> library(scaleboot)
> mam15.mt <- read.mt("mam15.mt")
> mam15.ass <- read.ass("mam15.ass")
> mam15.relltest <- relltest(mam15.mt,ass=mam15.ass)
> summary(mam15.relltest)
\end{Sinput}
\end{Schunk}

\subsection{Mam15 Dataset}

The results of the previous sections (
{\tt mam15.mt}, {\tt mam15.ass}, and
{\tt mam15.relltest}) are in fact stored in {\tt mam15} dataset of {\tt
scaleboot}. For users who want to try the examples, just type as follows.
\begin{Schunk}
\begin{Sinput}
> library(scaleboot)
> data(mam15)
\end{Sinput}
\end{Schunk}

The results for trees are extracted by
\begin{Schunk}
\begin{Sinput}
> mam15.trees <- mam15.relltest[1:15]
\end{Sinput}
\end{Schunk}

We have used a cluster computer of 40 cpus for parallel computing
using the {\tt snow} package.  The following code may take only 10
minutes, although we have used the number of resamples 10 times
larger than the default value.
\begin{Schunk}
\begin{Sinput}
> library(snow)
> cl <- makeCluster(40)
> library(scaleboot)
> mam15.mt <- read.mt("mam15.mt")
> mam15.ass <- read.ass("mam15.ass")
> mam15.relltest <- relltest(mam15.mt,nb=100000,ass=mam15.ass)
\end{Sinput}
\end{Schunk}

\subsection{Interpreting the Results}

First we sort the results in increasing order of log-likelihood difference,
\begin{Schunk}
\begin{Sinput}
> stat <- attr(mam15.trees,"stat")
> o <- order(stat)
> mam15.trees <- mam15.trees[o]
\end{Sinput}
\end{Schunk}
\begin{Schunk}
\begin{Sinput}
> summary(mam15.trees, k=1:3)
\end{Sinput}
\end{Schunk}
{\tiny
\begin{Schunk}
\begin{Soutput}
Corrected P-values by Akaike Weights Averaging (percent,Frequentist):
    raw          k.1          k.2          k.3          sk.1         sk.2         sk.3         beta0        beta1       hypothesis  model  weight 
t1  57.19 (0.16) 56.00 (0.05) 74.56 (0.06) 74.59 (0.08) 12.01 (0.09) 36.30 (0.09) 36.34 (0.10) -0.41 (0.00) 0.25 (0.00) alternative poly.3  54.75 
t3  32.26 (0.15) 30.33 (0.05) 46.53 (0.09) 45.42 (0.13) 60.66 (0.10) 79.58 (0.08) 78.79 (0.10)  0.30 (0.00) 0.21 (0.00) null        poly.3 100.00 
t2   3.72 (0.06)  3.69 (0.03) 12.84 (0.19) 15.75 (0.43)  7.39 (0.05) 20.44 (0.26) 23.71 (0.50)  1.46 (0.01) 0.33 (0.00) null        sing.3  99.97 
t5   1.34 (0.04)  1.39 (0.02)  8.38 (0.26) 11.19 (0.57)  2.79 (0.03) 12.72 (0.34) 15.92 (0.68)  1.79 (0.01) 0.41 (0.01) null        sing.3 100.00 
t6   3.14 (0.06)  3.15 (0.02) 12.71 (0.20) 14.83 (0.38)  6.30 (0.05) 19.84 (0.27) 22.22 (0.46)  1.50 (0.01) 0.36 (0.00) null        sing.3  79.76 
t7   0.52 (0.02)  0.53 (0.01)  3.33 (0.18)  3.99 (0.32)  1.05 (0.02)  5.19 (0.26)  6.01 (0.42)  2.20 (0.02) 0.36 (0.01) null        sing.3  77.61 
t4   1.52 (0.04)  1.50 (0.02) 10.00 (0.25) 13.19 (0.53)  3.00 (0.04) 14.89 (0.33) 18.42 (0.61)  1.73 (0.01) 0.44 (0.01) null        sing.3  65.91 
t15  0.08 (0.01)  0.08 (0.00)  1.20 (0.18)  1.96 (0.46)  0.15 (0.01)  1.77 (0.25)  2.69 (0.57)  2.71 (0.04) 0.45 (0.02) null        sing.3  81.81 
t8   0.00 (0.00)  0.00 (0.00)  0.06 (0.03)  0.14 (0.12)  0.00 (0.00)  0.09 (0.04)  0.18 (0.14)  3.71 (0.09) 0.44 (0.04) null        poly.3  71.94 
t14  0.22 (0.01)  0.22 (0.01)  2.40 (0.22)  3.72 (0.55)  0.44 (0.02)  3.59 (0.30)  5.16 (0.67)  2.41 (0.02) 0.43 (0.01) null        sing.3  84.14 
t13  0.02 (0.00)  0.01 (0.00)  0.50 (0.16)  1.26 (0.67)  0.03 (0.00)  0.70 (0.21)  1.58 (0.77)  3.13 (0.07) 0.53 (0.04) null        sing.3  65.03 
t9   0.00 (0.00)  0.00 (0.00)  0.05 (0.02)  0.07 (0.05)  0.00 (0.00)  0.07 (0.03)  0.10 (0.07)  3.75 (0.10) 0.43 (0.04) null        poly.3  63.50 
t11  0.00 (0.00)  0.00 (0.00)  0.04 (0.03)  0.21 (0.26)  0.00 (0.00)  0.06 (0.04)  0.25 (0.30)  3.95 (0.14) 0.57 (0.07) null        sing.3  71.34 
t10  0.00 (0.00)  0.00 (0.00)  0.00 (0.01)  0.01 (0.02)  0.00 (0.00)  0.01 (0.01)  0.01 (0.02)  4.57 (0.29) 0.65 (0.09) null        poly.3  84.06 
t12  0.00 (0.00)  0.00 (0.00)  0.00 (0.00)  0.00 (0.00)  0.00 (0.00)  0.00 (0.00)  0.00 (0.00)  5.40 (0.34) 0.39 (0.12) null        poly.2  56.42 
\end{Soutput}
\end{Schunk}
}

Next we look at the $p$-values. We confirm that $p_1$ (the second
column, indicated as {\tt k.1}) is almost the same as the raw BP (the first column); this
should be so if the model fitting is good. Only two trees, i.e., {\tt t1}
and {\tt t3}, have $p_1>0.05$. It is known that the bias of $p_1$ is large
so that often leads to false positives for tree selection. $p_2$
improves upon $p_1$ by correcting the bias. Six trees, i.e., {\tt t1}, {\tt t3},
{\tt t2}, {\tt t5}, {\tt t6}, and {\tt t4}, have $p_2>0.05$. $p_3$ improves upon $p_2$ even
more, although the trees of $p_3>0.05$ are the same six trees in this
example. In general, the accuracy of $p_k$ (AU $p$-values, indicated as {\tt k.k} in the table) increases for larger $k$, but it tends to have larger variation, so typically $k=2$ or $k=3$ would be used.
Terada and Shimodaira (2017) intoduced selective inference $p$-value, denoted as SI, or {\tt sk.k} in the table. In the phylogenetic tree selection, this may only apply to the tree selected by the algorithm. In our case, the maximum likelhiood tree ({\tt t1}), we can look at $p=0.36$ ({\tt sk.2}).  By taking this tree as the alternative hypothesis, we can interpret $1-p = 0.64$ as the selective $p$-value. So, we can reject the null ({\tt t1} is not true) and claim that {\tt t1} is the true tree when $p>0.95$ at $\alpha=0.05$. Unfortunately, $p=0.36$ is too small and we cannot say anything in this table.

Next we look at $p$-values for clusters instead of trees.
\begin{Schunk}
\begin{Sinput}
> mam15.edges <- mam15.relltest[16:25]  # 10 edges
> summary(mam15.edges,k=1:3)
\end{Sinput}
\end{Schunk}
{\tiny
\begin{Schunk}
\begin{Soutput}
Corrected P-values by Akaike Weights Averaging (percent,Frequentist):
    raw          k.1          k.2          k.3          sk.1         sk.2         sk.3         beta0        beta1       hypothesis  model  weight 
e1  58.53 (0.16) 57.94 (0.05) 71.83 (0.07) 72.23 (0.10) 15.89 (0.10) 33.75 (0.09) 34.15 (0.12) -0.39 (0.00) 0.19 (0.00) alternative poly.3 100.00 
e2  93.17 (0.08) 92.99 (0.04) 95.60 (0.07) 95.69 (0.08) 85.98 (0.07) 90.31 (0.12) 90.44 (0.15) -1.59 (0.00) 0.12 (0.00) alternative poly.3  86.39 
e3   3.74 (0.06)  3.68 (0.03) 12.38 (0.18) 16.88 (0.50)  7.36 (0.05) 19.83 (0.24) 24.87 (0.56)  1.47 (0.00) 0.32 (0.00) null        sing.3 100.00 
e4  32.85 (0.15) 31.77 (0.05) 43.46 (0.09) 43.31 (0.12) 63.54 (0.10) 77.41 (0.08) 77.29 (0.10)  0.32 (0.00) 0.15 (0.00) null        poly.3  98.19 
e5   6.00 (0.08)  5.96 (0.03)  7.49 (0.10)  7.24 (0.12) 11.92 (0.06) 14.31 (0.16) 13.97 (0.19)  1.50 (0.00) 0.06 (0.00) null        poly.3 100.00 
e6   1.75 (0.04)  1.75 (0.02)  6.94 (0.19) 11.13 (0.66)  3.50 (0.04) 11.14 (0.27) 16.02 (0.77)  1.79 (0.01) 0.31 (0.01) null        sing.3 100.00 
e7   3.66 (0.06)  3.76 (0.03)  9.13 (0.17) 10.52 (0.34)  7.52 (0.05) 15.52 (0.24) 17.20 (0.43)  1.56 (0.01) 0.22 (0.00) null        sing.3  99.99 
e8   0.00 (0.00)  0.00 (0.00)  0.04 (0.02)  0.06 (0.04)  0.01 (0.00)  0.06 (0.02)  0.09 (0.05)  3.69 (0.07) 0.31 (0.03) null        poly.3  65.37 
e9   0.30 (0.02)  0.30 (0.01)  1.38 (0.13)  2.02 (0.34)  0.60 (0.02)  2.28 (0.20)  3.11 (0.45)  2.48 (0.02) 0.27 (0.01) null        sing.3  92.20 
e10  0.00 (0.00)  0.00 (0.00)  0.02 (0.01)  0.08 (0.07)  0.00 (0.00)  0.02 (0.01)  0.10 (0.08)  4.09 (0.10) 0.48 (0.04) null        sing.3  72.06 
\end{Soutput}
\end{Schunk}
}
When we apply hypothesis testing to all the 10 clusters, we may use AU $p$-values. Let $p$ be {\tt k.2}. Then {\tt e8}, {\tt e9} and {\tt e10} have $p<0.05$, and {\tt e5}, {\tt e6}, {\tt e7} also have $p<0.10$. They may be rejected at either $\alpha=0.05$ or 0.10. However, people often want to look at only the clusters appearing in the top tree. In this table, only e1 and e2 are included in {\tt t1}. We should use SI $p$-values for testing each of these two clusters by taking it as the alternative hypothesis. Let $p$ be {\tt sk.2}. Then {\tt e2} has $p>0.90$, indicating that the null hypothesis ({\tt e2} is not true) is rejected as $1-p<0.10$ at $\alpha=0.10$.


Finally we examine model fitting.  According to the AIC values, the
fitting is good overall except for the top two trees; however note
that the AIC values should be about 10 times smaller if the default
value of nb=10,000 was used.  The fitting curves for the top four
trees are shown (Fig.~\ref{fig:mamplots}) by
\begin{Schunk}
\begin{Sinput}
> plot(mam15.trees[1:4])
\end{Sinput}
\end{Schunk}
\begin{figure}
\begin{center}
\includegraphics{usesb-047}
\caption{Model fitting for the top four trees}\label{fig:mamplots}
\end{center}
\end{figure}
According to the plots, the fitting is rather good even for the top two
trees. 

\begin{thebibliography}{9}

\bibitem{bib:Fels:85:CLP}
Felsenstein, J.
(1985).
\newblock Confidence limits on phylogenies: an approach using the bootstrap.
\newblock {\em Evolution} {\bf 39} 783--791.

\bibitem{bib:Garber2001}
  Garber, M. E. et al. (2001)
\newblock Diversity of gene expression in adenocarcinoma of the lung.
\newblock \emph{Proceedings of the National Academy of Sciences}
  {\bf 98} 13784--13789 (dataset is available from
  \url{http://genome-www.stanford.edu/lung_cancer/adeno/}).

\bibitem{bib:Shim:2002:AUT}
Shimodaira, H.
(2002).
\newblock An approximately unbiased test of phylogenetic tree selection.
\newblock {\em Syst. Biol.} {\bf 51} 492--508.

\bibitem{bib:Shimo:2008:TRN}
Shimodaira, H. (2008)
\newblock Testing Regions with Nonsmooth Boundaries via Multiscale Bootstrap.
\newblock \emph{Journal of Statistical Planning and Inference}
{\bf 138} 1227-1241
(\url{http://dx.doi.org/10.1016/j.jspi.2007.04.001}).

\bibitem{bib:SH1999} 
Shimodaira, H. and Hasegawa, M. (2001)
\newblock CONSEL: for assessing the
  confidence of phylogenetic tree selection.
\newblock \emph{Bioinformatics} {\bf 17} 1246--1247 (software is available from
  \url{http://stat.sys.i.kyoto-u.ac.jp/prog/consel/}).

\bibitem{bib:SS2006}
Suzuki, R. and Shimodaira, H. (2006)
\newblock pvclust: An R package for hierarchical clustering with $p$-values.
\newblock \emph{Bioinformatics} {\bf 22} 1540--1542 (software is available from
\url{http://stat.sys.i.kyoto-u.ac.jp/prog/pvclust/}).

\bibitem{bib:TS2017}
Terada, R. and Shimodaira, H. (2017)
\newblock Selective inference for the problem of regions via multiscale bootstrap.
\newblock \emph{arXiv:1711.00949}.

\bibitem{bib:Yang1997}
Yang, Z. (1997)
\newblock  PAML: a program package for phylogenetic analysis by
  maximum likelihood.
\newblock \emph{Computer Applications in BioSciences}
{\bf 13} 555--556 (software is available from
  \url{http://abacus.gene.ucl.ac.uk/software/paml.html}).


\end{thebibliography}

\end{document}
